\section*{Avant-propos}\addcontentsline{toc}{section}{Avant-propos}

\section{Avertissement}\label{avertissement}

L'ensemble de ce qui suit est le fruit de nombreuses heures
d'apprentissage sur Internet. En cherchant des solutions à des problèmes
concrets pour réaliser différentes opérations, j'ai accumulé de
nombreuse astuces qui facilitent aujourd'hui mon travail. Ne trouvant
pas une telle compilation déjà toute écrite (ce qui n'est pas étonné car
une combinaison d'utilisation est spécifique), j'ai décidé de les
rassembler dans un même document. Pour tout ce qui est informatique, je
suis autodidacte et je n'ai certainement pas des bases théoriques
suffisantes. Cela dit, je communique quotidiennement avec mon ordinateur
sous forme de lignes de commande. Ainsi, le document présent est le
résultat d'une approche très pragmatique et égo-centrée du terminal.
Néanmoins, je suis persuadé que beaucoup de matériel ici présenté
peut-être utile à beaucoup d'utilisateur même occasionnel du terminal.

\section{Mes sources}\label{mes-sources}

Il y a beaucoup de source d'information sur Internet mais je n'ai pas
trouvé une complilation d'usage multiple du terminal, alors je l'ai
fait, au moins pour moi. Voici quelques ressources qui vous seront, je
l'espère, utiles :

\begin{itemize}
\item
  Généralités
\item
  \href{https://fr.wikipedia.org/wiki/Terminal_informatique}{Terminal
  informatique}
\item
  \href{https://fr.wikipedia.org/wiki/Shell_Unix}{Shell Unix}
\item
  Base du Terminal

  \begin{itemize}
  \tightlist
  \item
    \href{http://www.graffitix.fr/index.php?pg=MXTeBT1}{graffitix}
  \end{itemize}
\item
  Liste des commandes

  \begin{itemize}
  \tightlist
  \item
    \href{http://ss64.com/bash/}{ss64}
  \end{itemize}
\item
  Astuces à la volée :

  \begin{itemize}
  \tightlist
  \item
    \href{http://www.maclife.com/search/terminal\%20101}{maclife}
  \item
    \href{http://www.magazine-avosmac.com/avosmacV4/}{A vos mac} (il
    faut être abonné).
  \item
    \href{http://osxdaily.com/}{OSX daily}
  \end{itemize}
\item
  Sites généralistes :

  \begin{itemize}
  \tightlist
  \item
    Regardez sur openclassroom (ancien site du zéro)
  \item
    Beaucoup d'astuces sur
    \href{http://www.commentcamarche.net/faq/4801-guide-d-utilisation-du-shell-pour-debutant}{Comment
    ça marche} (pour Linux mais beaucoup sont directement utilisables)
  \end{itemize}
\item
  Avancé :

  \begin{itemize}
  \tightlist
  \item
    \href{http://abs.traduc.org/abs-fr/ch01.html}{Faire des scripts
    bash}
  \item
    Le Bash est un langage commun aux systémes UNIX qui a une
    \href{https://www.gnu.org/software/bash/manual/bashref.html}{documentation
    officielle}.
  \end{itemize}
\end{itemize}

\section{Mes Abréviations}\label{mes-abruxe9viations}

Tout au long du document, j'emploie plusieurs abbréviations:

\begin{itemize}
\item
  Rq : Remarque
\item
  NB : Nota Bene
\item
  Ex : Exemple
\item
  ctrl : touche « ctrl~», touche «~contrôle~»
\item
  alt : touche «~alt~» aussi appelée touche option.
\item
  cmd : touche «~cmd~», «~commande~»
\item
  fn : touche «~fn~», touche «~fonction~»
\item
  tab : touche tabulation ~
\item
  espace : la barre d'espace
\item
  entr : la touche entrée
\item
  «~adr~» désigne un chemin et « adr/fichier1~», le chemin jusqu'au
  fichier \emph{fichier1} ; de même « adr/dossier~» désigne le chemin
  pour accéder à un dossier précis.
\item
  CPU : \emph{Central Processing Unit}, unité centrale de traitement.
\item
  GPU : \emph{Grpahics Processing Unit}, processeur graphique.
\end{itemize}

\chapter{\texorpdfstring{Le terminal et l'application
\emph{Terminal}}{Le terminal et l'application Terminal}}\label{le-terminal-et-lapplication-terminal}

\section{Sémantique}\label{suxe9mantique}

Le mot terminal fait référence au `terminal informatique' qui désigne la
partie d'un réseau informatique avec lequel un humain peut communiquer.
Cette communication consiste à lui faire exécuter une ou plusieurs
opérations. Aujourd'hui, pour la plupart des utilisateurs, le travail
avec l'ordinateur peut se faire de manière intuitive avec l'utilisation
d'une interface graphique, sans utiliser de ligne de commande. En raison
de l'utilisation de ligne de commande, les fenêtres d'invite de
commandes sont appelées `terminal'.

L'application terminal sous MacOSX L'application \emph{Terminal} est une
console pour de nombreux langages de programmation. Par défaut,
l'application lance le langage bash (Bourne-Again shell). On peut
remarquer ces 4 lettres dans la barre de titre. Ce langage permet
d'interagir avec son ordinateur à l'aide de lignes de commande.
L'utilisation du bash est le premier aspect du Terminal développé plus
bas. L'aspect minimaliste et un peu ésotérique d'un terminal dissimule
des possibilités immenses d'utilisation de son ordinateur. Concrètement,
au lieu d'utiliser diverses interfaces graphiques pour différentes
applications il est possible, au prix d'un effort d'apprentissage,
d'utiliser juste le \emph{Terminal}. Utiliser le \emph{Terminal} pour
regarder des photos, écouter de la musique ou naviguer sur Internet
n'est pas le plus évident. C'est tout de même possible et dans certains
cas, cela peut être d'une efficacité redoutable.

L'application \emph{Terminal} est une interface graphique possible et
native sous MacOSX, il en existe une seconde très poplaire :
\href{https://www.iterm2.com/index.html}{iTerm2}.

\href{http://invisible-island.net/xterm/xterm.html}{Xterm}

Sous Linix \href{https://doc.ubuntu-fr.org/terminal}{Le terminal
GNU/Linux}

https://doc.ubuntu-fr.org/terminal

Utilisez l'une ou l'autre est, éà mon sens, une question de préférence
en terme de faciliter dMutilisatuoib de design. Quand on utilise une
langage dans une fenêtre de terminal ce n'est pas le choix du terminal
qui change grand chose.

\section{Pourquoi utiliser le
terminal}\label{pourquoi-utiliser-le-terminal}

\begin{itemize}
\tightlist
\item
  on peut faire la plupart (même toutes) des actions sans, alors
  pourquoi ?
\item
  tout faire en dans une même fenêtre,
\item
  améliorer son workflow /
\item
  les emballages bien que utiles consomme des ressources.
\item
  parfois des actions qu'on crois possible qu'avec certains logiciels
  payants sont possibke pas avec des freware en ligne de commandes
\item
  ça ouvre un monde de software puissant qui ne sont pas toujours dans
  un emballage GUI améliorer notre literacy
\item
  invite à mieux conprendre son ordinateur
\end{itemize}

\section{Raccourcis clavier (par défaut) de l'application
terminal}\label{raccourcis-clavier-par-duxe9faut-de-lapplication-terminal}

La configuration des raccourcis peut se faire via l'édition d'un ficher
`.inputrc' dans le dossier utilisateur (à créer).

Par déafault on a Ces raccourcis sont valides pour le bash (le langage
pas défaut), mais aussi pour différentes applications. Cependant, ils
peuvent aussi être complètement occultés dans certains modes du
terminal. Ainsi, si vous utilisez l'éditeur de text \emph{nano} les
raccourcis ne sont plus valides (il y en a d'autres).

\section{Les racourcis de l'application
terminal}\label{les-racourcis-de-lapplication-terminal}

Il s'agit des raccourcis clavier par défault de MacOS 10.11.4 pour cette
application. Vous les retrouverez dans le menu de la console, je liste
ceux que je pense les plus utiles. Les raccourcis avec les cmd reste
valides quel que soit l'utilisation de la console. Ce n'est pas le cas
pour ceux qui utilisent la touche \textbf{ctrl} qui dépende de
l'utilisation ils sont valides au quand la console est utiliser pour
entrer des commandes Bash (mais parfois dans d'autres situations aussi).

\subsection{Gestion des fenêtres}\label{gestion-des-fenuxeatres}

\begin{itemize}
\tightlist
\item
  Créer une nouvelle fenêtre : \textbf{cmd + N}\\
\item
  Créer un nouvel onglet : \textbf{cmd + T}
\item
  Diviser la fenêtre du terminal : \textbf{cmd + D}
\item
  Annuler la division : \textbf{cmd + shift + D}
\item
  Efface le contenu d'un onglet : \textbf{cmd+K} (ou \textbf{ctrl+L}, ou
  utiliser la commande \textbf{clear})
\item
  Effacer la dernière ligne de commande et le(s) résultat(s) associé(s)
  : \textbf{cmd+L}
\item
  Faire une recherche dans le contenu de la console : \textbf{cmd+F}
\item
  Sauver le conteniu d'un onglet sous forme de text : \textbf{cmd+S}
\item
  Imprimer le contenu de la console : \textbf{cmd+P}
\end{itemize}

\subsection{Naviguer dans la console}\label{naviguer-dans-la-console}

En utilisant le terminal, il est fréquent d'accumuler bien plus de
lignes dépassant largement la hauteur de la fenêtre du terminal, afin de
naviguer dans le contenu d'un onglet on peut utiliser la barre de
défilement, mais aussi :

\begin{itemize}
\tightlist
\item
  Remonter à la première ligne : \textbf{fn + flèche de gauche}
\item
  Descendre à la dernière ligne : \textbf{fn + flèche de gauche}
\item
  Remonter d'une page : \textbf{fn + flèche de gauche}
\item
  Descendre d'une page : \textbf{fn + flèche de gauche}
\end{itemize}

\subsection{Naviguer au sein d'une ligne de
commande}\label{naviguer-au-sein-dune-ligne-de-commande}

\begin{itemize}
\tightlist
\item
  Avancer d'un symbole : \textbf{flèche de droite} (ou \textbf{ctrl+p})
\item
  Reculer d'un symbole : \textbf{flèche de gauche} (ou \textbf{ctrl+b})
\item
  Avancer d'un symbole : \textbf{alt + flèche de droite}
\item
  Avancer d'un symbole : \textbf{alt + flèche de droite}
\item
  Aller au début d'une ligne : \textbf{ctrl + A}\\
\item
  Aller à la fin d'une ligne : \textbf{ctrl + E}\\
\item
  Aller d'un bout de la ligne à l'autre : \textbf{ctrl + X} (2 fois)
\end{itemize}

\subsection{Ecrire efficacement une ligne de
commande}\label{ecrire-efficacement-une-ligne-de-commande}

\begin{itemize}
\tightlist
\item
  Auto-complétion: \textbf{tab} ; l'auto-complétion est l'action de
  complèter un mot / une commande dont on a entré les premiers
  charactères.
\item
  Si l'auto-complétion ne fonctionne pas deux situations sont à
  envisager :

  \begin{enumerate}
  \def\labelenumi{\arabic{enumi}.}
  \tightlist
  \item
    il n'y a pas de correspondance avec un nom de fichier / de commande
    / \ldots{} existant,
  \item
    plusieurs choix sont disponibles, dans ce cas, en utilisant
    \textbf{tab} de nouveau, on affiche la liste des possibilités.
  \end{enumerate}
\item
  Effacer le caractère à gauche du curseur : \textbf{backspace} (le
  classique) ou \textbf{ctrl + H}
\item
  Effacer le caractère à droite du curseur : \textbf{ctrl + D} attention
  si utilisé sur une ligne vite, entraîne la déconnection)
\item
  Effacer tout ce qui se trouve à gauche du curseur : \textbf{ctrl + U}
\item
  Effacer tout ce qui se trouve à droite du curseur : \textbf{ctrl + K}
\item
  Effacer le mot juste à gauche du curseur : \textbf{ctrl + W}
\item
  Inverser les deux dernières lettres : \textbf{ctrl + T}
\item
  Coller la saisie précédente : \textbf{ctrl + Y}
\item
  Rappeler la dernière commande : \textbf{flèche du haut} (ou
  \textbf{ctrl+P})
\item
  Revenir vers une commande plus récente : \textbf{flèche du bas} (ou
  \textbf{ctrl+N})
\item
  Recherche dans l'historique : \textbf{ctrl + R}, il suffit alors
  d'entrer les premières charactère de la ligne ayant été utilisée
  précedemment.
\end{itemize}

\subsection{Executer/arrêter une ligne de
commande}\label{executerarruxeater-une-ligne-de-commande}

\begin{itemize}
\tightlist
\item
  Exécuter une commande : écrire la commande puis \textbf{entr} (ou
  \textbf{ctrl+J} ou )
\item
  Annuler l'exécution d'une commande : \textbf{ctrl + C}
\item
  Stopper une tâche : \textbf{ctrl + Z}
\item
  Déconnection de la session terminal: \textbf{ctrl + D} (sur une ligne
  vide)
\end{itemize}

\textbf{NB} : Pour ne pas entrer manuellement un chemin, on peut faire
glisser le dossier ou le fichier concerné, son chemin apparaîtra alors.

\section{Les commandes ?}\label{les-commandes}

De manière générale, pour un langage dooné, une commande est un ensemble
de charactères interprété par l'ordinateur comme une instruction
induisant la réalisation d'une tâche par celui-ci (si la commande est
correctement entrée). Quelques spécifications pour le langage bash :

\begin{itemize}
\tightlist
\item
  Utilisation d'une commande. Dans «~Terminal~», il est nécessaire
  d'entrer le nom de la commande puis éventuellement l'option qui est en
  général un tiret (``-'') et une ou ou deux lettres puis éventuellement
  des paramètres suivis éventuellement de l'objet qu'elle concerne
  (souvent un fichier représenter par son chemin) : nomcommande -option1
  -option2 paramètre objet (il suffit alors de taper \textbf{entr}). Les
  options peuvent être mises les unes à la suite des autres
  -option1option2.
\end{itemize}

Exemple 1, une commande sans option

\begin{Shaded}
\begin{Highlighting}[]
  \KeywordTok{ls}
\end{Highlighting}
\end{Shaded}

le résulat est une liste de fichiers et de dossier là où vous êtes

Il n'y a que la commande \emph{ls} qui liste les fichiers du repertoire
courant, - Ex 2 :

\begin{Shaded}
\begin{Highlighting}[]
  \KeywordTok{ls} \NormalTok{-a}
\end{Highlighting}
\end{Shaded}

la même commande avec une option -a (afficher tous les fichiers, même
les fichiers cachés),

\begin{verbatim}
- Ex 3 :

> ls -a ~/Documents
\end{verbatim}

je rajoute le répertoire pour lequel je veux la listes de tous les
fichiers

\begin{verbatim}
- Ex 4 :

> mkfile -nv 100k ~/Desktop/exemple1.txt
\end{verbatim}

Crée un fichier avec deux options \emph{-n} et \emph{-v} (raccourcis en
\emph{-nv}), le paramètre 100k (taille du fichier) et le nom du fichier
\emph{\textasciitilde{}/Desktop/exemple1.txt}, L'adresse d'un fichier
qui va être créé.

\begin{itemize}
\item
  Une commande est en fait un exécutable dont on peut facilement trouver
  l'adresse : \textbf{which + nom\_commande}, une recherche large est
  possible avec : \textbf{locate + nom\_commande}
\item
  Les commandes ne sont pas sensible à la casse. ex : ls = LS = lS = Ls.
  \emph{Attention!!} ce n'est pas le cas pour les options.
\item
  Pour tout renseignement relatif è une commande (notamment pour en
  connaître les options) : \textbf{man + nom\_commande} (\textbf{q} pour
  quitter le manuel), celui-ci décrit la fonction de la commande. Pour
  savoir comment faire défiler l'écran, une fois dans dans le mode
  manuel, il faut entrer h (en bref on peut utiliser les flèches, les
  flêches avec \textbf{fn} (ou encore \textbf{b }et \textbf{z} ou
  \textbf{espace} également).
\item
  Parfois un droit d'administrateur est requis pour certaines actions,
  typiquement lorsqu'un message ``permission denied'' apparaît suite à
  l'entrée de la commande. Pour pouvoir exécuter la commande il faut
  avoir les droits d'administrateurs et faire précéder la commande de :
  \textbf{sudo} («~super user do~»).Le mot de passe associé au compte
  administrateur est alors demandé.
\end{itemize}

\subsection{\texorpdfstring{Les commandes basiques du
\emph{bash}}{Les commandes basiques du bash}}\label{les-commandes-basiques-du-bash}

Dans la suite je présente différentes commandes du bash. Je ne
détaillerai que très peu d'options, pour les connaître, utiliser le
manuel. Dans le \emph{bash}, le charactère \# est utilisé pour
introduire des commentaires.

\subsubsection{Naviguer à travers les différents
répertoires}\label{naviguer-uxe0-travers-les-diffuxe9rents-ruxe9pertoires}

\begin{itemize}
\tightlist
\item
  Changer de répertoire :
\end{itemize}

\begin{quote}
cd adr\\
\# Exemple :\\
cd /dossiera/dossierb/dossierC/file1.txt
\end{quote}

\begin{itemize}
\item
  quelques cas particuliers :

  \begin{quote}
  \# Aller à la racine :\\
  cd /\\
  \# Aller au dossier utilisateur :\\
  cd \textasciitilde{}\\
  \# Aller au dossier supérieur :\\
  cd ..
  \end{quote}
\item
  Connaître le répertoire actuel :
\end{itemize}

\begin{quote}
pwd
\end{quote}

\begin{itemize}
\tightlist
\item
  Lister les fichiers d'un dossier :
\end{itemize}

\begin{quote}
ls adr
\end{quote}

si aucun nom de dossier n'est donné, ce sont les fichiers du répertoire
courant qui sont listés. Option \emph{-a} pour voir les fichiers cachés
et option \emph{-l} pour obtenir des informations sur les fichiers (date
de modification et qui est propriétaire).

\begin{quote}
ls -alth
\end{quote}

\begin{itemize}
\tightlist
\item
  Connaître la taille des fichiers
\end{itemize}

\begin{quote}
du adr
\end{quote}

option \emph{-s} pour faire la somme de la taille des fichiers d'un
dossier et option \emph{-h} pour avoir des nombres en unité facile à
lire. ex : du -sh \textasciitilde{}/Documents

\begin{itemize}
\tightlist
\item
  Créer un nouveau fichier :
\end{itemize}

\begin{quote}
mkfile + adr/nom\_fichier
\end{quote}

On peut préciser la taille du fichier ex: \textgreater{} mkfile 140k
truc.txt

\begin{itemize}
\tightlist
\item
  Créer un nouveau dossier :
\end{itemize}

\begin{quote}
mkdir + adr/nom\_dossier
\end{quote}

\begin{itemize}
\tightlist
\item
  Copier un fichier :
\end{itemize}

\begin{quote}
cp adr1/fichier1 adr2/fichier2
\end{quote}

Si fichier2 n'est pas préciser, fichier1 est utilisé comme nom de
fichier.

\begin{itemize}
\tightlist
\item
  Copier un dossier :
\end{itemize}

\begin{quote}
cp -R adr1/dossier1 adr2/dossier2
\end{quote}

\begin{itemize}
\tightlist
\item
  Déplacer un fichier ou un dossier :
\end{itemize}

\begin{quote}
mv adr1/fichier1 adr2/fichier2
\end{quote}

\begin{itemize}
\tightlist
\item
  Suppimer un fichier :
\end{itemize}

\begin{quote}
rm +adr/fichier
\end{quote}

\begin{itemize}
\tightlist
\item
  Suppimer un dossier :
\end{itemize}

\begin{quote}
rm -R adr/dossier (option f pour forcer) rmdir adr/dossier
\end{quote}

\begin{itemize}
\tightlist
\item
  Vider la corbeille :
\end{itemize}

\begin{quote}
~rm -r \textasciitilde{} /.Trash/*
\end{quote}

\begin{itemize}
\tightlist
\item
  Ouvrir un fichier :
\end{itemize}

\begin{quote}
open + adr/fichier
\end{quote}

Pour visualiser un fichier en lecture seule on utilise la commande less
:

option \emph{-a} permet de spécifier l'application avec laquelle pouvrir
le fichier, ex: open -a Finder / )

\begin{itemize}
\tightlist
\item
  Changer le propriétaire d'un fichier :
\end{itemize}

\begin{quote}
chown owner{[}:group{]} adr/fichier
\end{quote}

Défini à qui appartient le fichier.

\begin{itemize}
\tightlist
\item
  Changer les droits d'accès d'un fichier :
\end{itemize}

\begin{quote}
chmod mode adr/fichier
\end{quote}

Les modes permettent de définir les actions possibles sur un fichiers
pour tous types d'utilisateurs, propriétaire ou non du fichier. Les
modes s'écrivent sous forme d'un nombre précisant les droits des
utilisateurs sur un fichier.

\begin{itemize}
\tightlist
\item
  Imprimer un fichier :
\end{itemize}

\begin{quote}
lpr adr/fichier
\end{quote}

Attention, tous les types de fichiers ne sont pas supportés.

\section{Éteindre/redémarrer et relancer le
Finder}\label{uxe9teindrereduxe9marrer-et-relancer-le-finder}

\begin{itemize}
\tightlist
\item
  Eteindre :
\end{itemize}

\begin{quote}
sudo shutdown -h now
\end{quote}

\begin{itemize}
\tightlist
\item
  Eteindre dans 30 minutes :
\end{itemize}

\begin{quote}
sudo shutdown -h +30
\end{quote}

\begin{itemize}
\tightlist
\item
  Eteindre le 06/12/2015 à 15h30 :
\end{itemize}

\begin{quote}
sudo shutdown -h 1512061530
\end{quote}

\begin{itemize}
\tightlist
\item
  Redémarrer :
\end{itemize}

\begin{quote}
sudo shutdown -r now\\
\# plus simplement :\\
sudo reboot
\end{quote}

\begin{itemize}
\tightlist
\item
  Relancer le Finder :
\end{itemize}

\begin{quote}
sudo killall Finder
\end{quote}

\section{Gérer l'ordinateur}\label{guxe9rer-lordinateur}

\subsection{La date}\label{la-date}

\begin{itemize}
\tightlist
\item
  Obtenir la date :
\end{itemize}

\begin{quote}
date
\end{quote}

\begin{itemize}
\tightlist
\item
  Changer la date :
\end{itemize}

\begin{quote}
date 0613162785
\end{quote}

\begin{itemize}
\tightlist
\item
  Changer la date :
\end{itemize}

\begin{quote}
date 1758
\end{quote}

\subsection{Veille de l'ordinateur et suspensions des
executions}\label{veille-de-lordinateur-et-suspensions-des-executions}

\begin{itemize}
\item
  Laisser l'ordinateur en activité (pas de veille) :

  \begin{quote}
  caffeinate
  \end{quote}

  option \emph{-d} pour que l'écran ne se mette pas en veille, option
  \emph{-t} pour choisir le temps (en seconde). Ex : caffeinate -dt 600
  =\textgreater{} permet de maintenir l'ordinateur mais aussi l'écran en
  activité pendant 10 minutes.
\item
  Suspendre les exécutions pendant une durée «~tps~» exprimé en secondes
  :

  \begin{quote}
  sleep + tps
  \end{quote}

  Ex : si j'utilise ``sleep 10'' et que j'essaye ensuite d'entrer la
  commande ls, je devrais attendre la fin des des 10 secondes pour que
  la commande soit exécutée.
\end{itemize}

\subsection{Visualiser l'acitivité de son
ordinateur}\label{visualiser-lacitivituxe9-de-son-ordinateur}

\begin{itemize}
\tightlist
\item
  Visualiser l'activité :
\end{itemize}

\begin{quote}
top
\end{quote}

Cette commande affiche un grand tableau, la table des processus (Table
Of Processes). Un processus est une tâche entreprise par l'ordinateur
qui est répertorié tant qu'elle a court. Parmi les colonnes, on a : 1.
\textbf{PID} : le numéro d'identité du processus 2. \textbf{COMMAND} :
le nom du processus 3. \textbf{\%CPU} : le pourcentage CPU utilisé, il
s'agit d'une mesure unstantannée de la charge de travail du processeur
(on a jusqu'à n*100\% de CPU disponible ou n est le nombre de coeur.).
4. \textbf{\%TIME} : la durée d'activité du processus 4. \textbf{MEM} :
la mémoire utilisée par le processus

Une fois que l'on affiche le tableau, le terminal fonctionne un peu
différemment. Pour connaître ce qu'on peut faire, il suffit d'entrer
\textbf{?}. Une action pratique est de trier le tableau, il fa=ut alors
taper \textbf{o} puis le nom de la colonne, par exemple \textbf{cpu}
pour afficher par utilisation de cpu et \textbf{cpu} pour trier par
quantité de mémoire utilisée. Pour quitter le tableau, entrer
\textbf{q}.

\begin{itemize}
\tightlist
\item
  Arrêter une tâche :
\end{itemize}

\begin{quote}
kill + Pid
\end{quote}

Le Pid est le numéro d'indentité donné par le taleau top.

\begin{quote}
kill -STOP Pid kill -COUNT Pid
\end{quote}

\href{http://osxdaily.com/2013/05/30/pause-resume-app-process-mac-os-x/}{kill}

\begin{itemize}
\tightlist
\item
  Connaître son «load average~» :
\end{itemize}

\begin{quote}
uptime
\end{quote}

Le load average désigne, sous les systèmes UNIX, une moyenne de la
charge système, une mesure de la quantité de travail que fait le système
durant la période considérée. (cf.~Wikipedia)

\begin{itemize}
\item
  Exécuter des tâches en arrière plan :
\item
  Chaque commande peut être exécutée en arrière plan, pour cela on
  ajoute \& à la fin de celle-ci. L'identité «~pid~» du processus est
  alors donné. Il s'agit d'un numéro de tâche qui s'inscrivent à chauqe
  nouvelle tâche et cela, depuis que vous avez redémarrez votre
  ordinateur. Ce numéro est accessible grâce à la commande top, pour le
  supprimer : kill pid
\item
  Ex : sleep 10\& (cette fois vous pourrez exécuter ls !)
\item
  Pour faire revenir le processus au premier plan : fg \%njob (premier
  numéro sur la gauche donné par la commande jobs)
\item
  Un processus stoppé peut être passé à l'arrière plan : bg \%njob
\item
  Avec top, on peut avoir accès aux identités des processus : kill pid,
  supprime le processus ; kill -stop pid le met en pause, kill -cont pid
  le réactive.
\end{itemize}

\subsection{Disques et clef USB}\label{disques-et-clef-usb}

\begin{itemize}
\item
  Forcer l'éjection d'un cd : \textgreater{} sudo drutil eject
\item
  Liste des volumes montés et leurs caractéristiques : df (-h
  «human~»pour avoir des préfixes connus, \emph{~ pour l'ensemble des
  dossiers d'un dossier (avec un espace) ! ex : du -sh }
  \textasciitilde{})
\end{itemize}

\subsection{Réseaux}\label{ruxe9seaux}

\begin{itemize}
\tightlist
\item
  Afficher l'ensemble des disques utilisés :
\end{itemize}

\begin{quote}
diskutil list
\end{quote}

\begin{itemize}
\tightlist
\item
  Visualiser/configurer les réseaux :
\end{itemize}

\begin{quote}
ifconfig
\end{quote}

\begin{itemize}
\tightlist
\item
  Obtenir son adresse IP :
\end{itemize}

\begin{quote}
ipconfig getifaddr en0
\end{quote}

ou en1 si «~en2~» dans ifconfig existe

\begin{itemize}
\tightlist
\item
  Regarder l'état des réseaux que l'on utilise :
\end{itemize}

\begin{quote}
netstat netstat -A
\end{quote}

\subsection{Astuce Apple}\label{astuce-apple}

\begin{itemize}
\tightlist
\item
  Changer le type du fichier de capture d'écran (.png par défaut):
\end{itemize}

\begin{quote}
defaults write com.apple.screencapture type PDF (ou jpg ou png)
\end{quote}

\begin{itemize}
\tightlist
\item
  Afficher les fichiers cachés defaults :
\end{itemize}

\begin{quote}
write com.apple.finder AppleShowAllFiles 1
\end{quote}

Remplacer le 1 par 0 pour les cacher, pour que la manipulation soit
effective, il faut relancer le Finder.

\begin{itemize}
\tightlist
\item
  Reconstruire reconstruire l'index du spotlight :
\end{itemize}

\begin{quote}
sudo mdutil -E /
\end{quote}

\begin{itemize}
\tightlist
\item
  Reconstruire reconstruire l'index du spotlight d'un volume :
\end{itemize}

\begin{quote}
sudo mdutil -E /Volumes/{[}DriveName{]}
\end{quote}

\subsection{Créer une archive :}\label{cruxe9er-une-archive}

\begin{itemize}
\tightlist
\item
  Créer une archive tar : tar adr/fichier (beaucoup de possibilité,
  ajouter dans une archive\ldots{}, à développer)
\item
  Créer une archive zip : gzip adr/fichier (option -k pour garder la
  copie non zippée)
\item
  Dezipper une archive zip : gunzip adr/fichier
\end{itemize}

\subsection{Connaître son Hardware :}\label{connauxeetre-son-hardware}

\begin{itemize}
\tightlist
\item
  system\_profiler SPHardwareDataType
\item
  sysctl hw
\item
  sysctl machdep.cpu
\end{itemize}

\subsection{Fichiers cachés}\label{fichiers-cachuxe9s}

\begin{itemize}
\item
  Rq : Pour créer rapidement un fichier caché, il suffit de faire
  commencer son nom par un «~.~».
\item
  Faire d'un fichier caché, un fichier apparent : sudo chflags nohidden
  nomdufichier (option -R pour un dossier)
\end{itemize}

\subsection{Recherche de fichiers}\label{recherche-de-fichiers}

Nous sommes souvent en train de rechercher des fichiers et bien que
«~Spotlight~» soit souvent d'une grande aide il est parfois difficile
d'avoir exactement ce que l'on veut. Pour des recherches très efficace,
il existe des commandes bash d'une très grande efficacité mais qui
demandent un peu d'entrainement !

\subsubsection{expressions
régulières}\label{expressions-ruxe9guliuxe8res}

grep grep -rwl niche \textasciitilde{}/Desktop/interactionLS/ options :
-r dans le subfolder -w le mot entier -l liste de fichier -v l'inverse
de ce qui est chercher grep .tex grep \^{}sep.r.te /usr/share/dict/words

\begin{itemize}
\item
  Recherche dans l'historique : history \textbar{} grep gcc
\item
  Liste des expressions régulières : ------------------ \^{} debut \$
  fin ------------------
\item
  plusieurs fois le caractère précédent (au moins une fois)
\item
  zéro ou plusieurs fois le caractère précédent ? zéro ou une seule fois
  le caractère précédent \{x\} x fois le caractère précédent \{x,\} x
  fois ou plus le caractère précédent \{i,j\} minimum i fois, maximum j
  fois le caractère précédent ------------------ ( ) ces règles peuvent
  s'appliquer sur un ensemble de caractères/expression entre parenthèse
  ------------------ \textbar{} ou . n'importe quel caractère une seule
  fois ------------------ {[} {]} liste de caractères autorisés sur
  lesquels peuvent s'appliquer les règles précédentes {[}\^{} {]} liste
  de caractères interdits
\item
  permet de définir des intervalles {[}a-z{]}{[}:alpha:{]} ou
  {[}a-zA-Z{]}{[}:digit:{]} ou {[}0-9{]}{[}:album:{]} ou
  {[}a-zA-Z0-9{]}{[}:blank:{]} {[}:punct:{]} caractère de ponctuation
  ------------------
\end{itemize}

\subsection{chercher un fichier}\label{chercher-un-fichier}

\begin{quote}
find \textasciitilde{}/ -iname `evol' find \textasciitilde{}/Desktop
-iname `*.txt'
\end{quote}

\begin{itemize}
\tightlist
\item
  Trouver les différences entre des fichiers :
\end{itemize}

\begin{quote}
vimdiff file1 file2 file3
\end{quote}

\begin{itemize}
\tightlist
\item
  Réunir des textes en un seul :
\end{itemize}

\begin{quote}
cat sample1.txt sample2.txt sample3.txt \textgreater{} sample-all.txt
\end{quote}

\begin{itemize}
\tightlist
\item
  metadata search : \textgreater{} mdfind word1 word2 word3
\end{itemize}

ex : mdfind wavelet ecology markov test fourier royal society chavez
using or (royal or society)

\begin{quote}
mdfind wavelet ecology markov test fourier (royal \textbar{} society)
mdfind -onlyin \textasciitilde{}/Movies couleur
\end{quote}

\subsection{Créer un alias}\label{cruxe9er-un-alias}

\begin{itemize}
\tightlist
\item
  Lorsqu'on utilise à répétition une commande, il peut être intéressant
  de créer un alias. Par exemple, j'ai souvent besoin d'aller dans mon
  dossier de thèse : cd \textasciitilde{}/Documents/PHD, j'aimerais
  simplement taper «phd», pour cela : alias php `cd
  \textasciitilde{}/Documents/PHD'
\item
  De manière générale le principe est le suivant : alias
  +nom\_de\_la\_commade\_désirée +~`la commande entre guillemets'
\item
  La procédure précédente permet seulement d'avoir une commande pendant
  la session, pour la garder en mémoire il est nécessaire de
  l'enregistrer dans le fichier «./bash\_profile~» dans le dossier
  utilisateur (\textasciitilde{}). Une procédure possible est alors : 1-
  cd \textasciitilde{} 2- nano \textasciitilde{}/bash\_profile 3- on
  ajoute la ligne : alias php `cd \textasciitilde{}/Documents/PHD' 4- on
  enregistre et voila ! Il faudra alors soit rouvrir un onglet, relancer
  le terminal ou entrer : source \textasciitilde{}/bash\_profile pour
  que ce soit ok
\item
  Si par hasard, on souhaite créer un alias qui a un nom qui existe déjà
  (typiquement pour ne pas rentrer une option à chaque fois),
  l'antislash «~~» rétabli la commande par défaut. Ex : si j'utilise
  très fréquemment ls -la tout le temps =\textgreater{} je peux alors
  créer l'alias : alias ls `ls -la' et alors pour avoir le ls d'origine
  =\textgreater{} la commande : \textbackslash{}ls est alors le ls par
  défaut.
\end{itemize}

screen screen -d
http://www.tecmint.com/screen-command-examples-to-manage-linux-terminals/

\chapter{Les gestionnaires de paquets sous
MacOS}\label{les-gestionnaires-de-paquets-sous-macos}

Il s'agit d'application qui facilitent l'installation de logiciels
libres. Ces différents logiciels sont installés facilement en quelques
lignes de commandes. C'est une pratique qui existe par défaut dans les
différentes distributions de Linux (apt-get, yum,\ldots{}) mais qui doit
être installé pour MacOS. Il existe plusieurs gestionnaires de paquets :
Fink, MacPort et Homebrew. Je présente MacPort et Homebrew dans les
sections qui suivent. Je n'utilise plus MacPort depuis mi-2014, je suis
passé à Homebrew, il se peut donc que certaines indications ne soient
pas correctes. Une installtion propore

\section{MacPorts}\label{macports}

Avec MacPorts, tout ce que vous demandez sera installé dans le
répertoire : /opt/local. Pour voir la liste des exécutables (les
logiciels libres) qui seront appelés par une commande le plus souvent de
ce même nom : ls /opt/local/bin

\subsection{Installation de MacPorts}\label{installation-de-macports}

\begin{itemize}
\tightlist
\item
  Il faut commencer par l'installer : http://www.macports.org/
\item
  Toutes (très très complet!) les indications sont à l'adresse suivant :
  http://guide.macports.org/
\item
  Pour apprendre :
  http://fr.openclassrooms.com/informatique/cours/utilisez-macports
\item
  NB : Il faut veiller à avoir la dernière version de «~Xcode~» (ça
  change entre les OS ! télécharger toujours la dernière version de
  Xcode adaptée à l'OS) : puis installer le guide d'outil du développeur
  avec la commande : xcode-select --install
\end{itemize}

\subsection{Utiliser MacPorts}\label{utiliser-macports}

\begin{itemize}
\item
  Pour entrer en mode actif : port
\item
  Pour entrer avec les droits admin : sudo port
\item
  Sinon on met port (ou sudo port) en début de commande (on entre pas
  dans le mode actif) (option installed, uninstalled)
\item
  Voir la liste des packages : list
\item
  Chercher avec un mot clef : search + mot
\item
  Des infos sur un port : info + nom du port
\item
  Les dépendances d'un port : info + nom du port
\item
  Installer des packages/ mises à jour :
\item
  Attention pour pouvoir installer un package, il faut avoir les droits
  d'administrateur et donc il est nécessaire de se connecter en «~sudo
  port~»
\item
  Installer un package : install +nom package
\item
  Désinstaller un package : uninstall +nom package (option
  ---follow-dependents, pour désinstaller les dépendances)
\item
  Mise à jour de la liste de package : sync
\item
  Port pas à jour : port outdated (on peut ajouter list pour un
  défilement : port list outdated ) /etc/macports/sources.conf,
\item
  Mise à jour de la liste et des packages installés : selfupdate
\item
  NB : si le message d'erreur suivant apparaît : Error: Error
  synchronizing MacPorts sources: command execution failed cela est
  vraisemblablement du à votre connexion (cela m'arrive quand je suis au
  labratoire!).
\item
  Mise à jour d'un seul package : upgrade + nom package
\item
  Mise à jour des ports pas à jour : sudo port upgrade outdated
\item
  Chercher port qui dépendent d'un nom donné : port echo
  depends:nompackage
\item
  Cleaning package (je suis pas bien capable de vous dire ce que ça
  fait) : port clean --- all installed ; port clean --- dist
\item
  Par défaut les versions anciennes, inactives (donc plus requises) ne
  sont pas enlever, pour le faire : sudo port -v uninstall inactive
\item
  J'ai eu pas mal de problème pour synchroniser il semble que parfois on
  rencontre des problèmes avec Xcode (encore une fois vérifier que vous
  avez bien la dernière version, la solution est à nouveau sur le site
  de Macports)\ldots{}
\item
  Pour régler le problème de synchronisation que j'ai rencontré, il
  suffit de ne pas prendre la voie courante, on peut alors aller dans le
  fichier «~mdfind source.conf~» (où est-il ? =\textgreater{} mdfind
  -name sources.conf), il faut commenter la dernière ligne «rsync
  \ldots{}~» et ajouter en-dessous:
  https://distfiles.macports.org/ports.tar.gz //To use the rsyncd server
  you must copy /opt/local/etc/rsyncd.conf.example to rsyncd.conf and
  add your modules there. //See `man rsyncd.conf' for more information
\item
  Désinstallation : enlever tous les packages: sudo port -fp uninstall
  installed
\item
  Enlever toute trace de MacPort: sudo rm -rf\\
  /opt/local\\
  /Applications/DarwinPorts\\
  /Applications/MacPorts\\
  /Library/LaunchDaemons/org.macports.*\\
  /Library/Receipts/DarwinPorts\emph{.pkg\\
  /Library/Receipts/MacPorts}.pkg\\
  /Library/StartupItems/DarwinPortsStartup\\
  /Library/Tcl/darwinports1.0\\
  /Library/Tcl/macports1.0\\
  \textasciitilde{}/.macports
\item
  Encore une fois le guide officiel est complet: guide officiel
\end{itemize}

\section{Homebrew}\label{homebrew}

\href{http://brew.sh}{Homebrew} est, à mon sens, encore plus facile
d'utilisation et très prometteur pour les années à venir (il y a de
grandes facilités pour les dévreloppeurs). Homebrew est aussi un
gestionnaire de package qui est équivalent MacPort, le vocaublaire
employé (les commandes) font écho au homebrew, les noms des paquets
deviennent des formules/recette que l'on met à jour et que l'on
brasse/remue (brew) pour les installer/mettre à jour sur notre
ordinateurs. Visitez
\href{https://zestedesavoir.com/tutoriels/578/homebrew-pour-gerer-ses-logiciels-os-x/}{OpenClassroom}
pour des informations complémentaires. les paquets seront stockés dans
/usr/local/Cellar.

\subsection{Installation et
désinstallation}\label{installation-et-duxe9sinstallation}

Avant d'installer MacPort vérifier que Xcode soit installé, que vous
ayez accepté les closes de la license et d'avoir entrer la ligne de
commande suivante : \textgreater{} xcode-select --install

Installation, entrez dans le terminal :

\begin{quote}
ruby -e ``\$(curl -fsSL
https://raw.githubusercontent.com/Homebrew/install/master/install)''
\end{quote}

Désinstallation :

\begin{quote}
/usr/bin/ruby -e ``\$(curl -fsSL
https://raw.githubusercontent.com/Homebrew/install/master/uninstall)''
\end{quote}

\subsection{Commande principales}\label{commande-principales}

brew update brew upgrade brew remove brew list brew doctor brew prune
brew tape brew info brew update; brew upgrade; brew cleanup; brew cask
cleanup;

\subsection{Mon installation}\label{mon-installation}

\begin{quote}
brew install gdal geos proj r imagemagick ffmpeg postgresql pandoc
pandoc-citeproc valgrind
\end{quote}

NB : cask est une extension de brew il suffit d'utiliser brew cask ou
mettre le chemin complet:

\begin{quote}
brew cask install julia mactex dropbox copy rstudio atom filezilla
libreoffice gimp inkscape mendeley-desktop xquartz appcleaner vlc
\end{quote}

\begin{quote}
brew tap homebrew/science
\end{quote}

\begin{quote}
brew install r imagemagick ffmpeg postgresql lynx
\end{quote}

NB : cask est une extension de brew il suffit d'utiliser brew cask ou
mettre le chemin complet:

\begin{quote}
brew cask install java julia mactex dropbox copy rstudio brew install
Caskroom/cask/libreoffice Caskroom/cask/appcleaner Caskroom/cask/atom
Caskroom/cask/xquartz Caskroom/cask/filezilla
\end{quote}

\begin{quote}
sudo chown -R \$(whoami):admin /usr/local
\end{quote}

\chapter{Developpement web}\label{developpement-web}

MacOS 10.10
http://ole.michelsen.dk/blog/setup-local-web-server-apache-php-osx-yosemite.html
MacOS 10.11
http://coolestguidesontheplanet.com/get-apache-mysql-php-and-phpmyadmin-working-on-osx-10-11-el-capitan/

sudo apachectl start sudo apachectl restart httpd -v

Setup local web server with Apache and PHP on OS X El capitan

sudo nano /etc/apache2/users/username.conf

\chapter{Des logiciels libres dans le
terminal}\label{des-logiciels-libres-dans-le-terminal}

\chapter{Git}\label{git}

\begin{quote}
git config --global user.name ``KevCaz'' git config --global user.email
kevin.cazelles@gmail.com
\end{quote}

\begin{quote}
git reset
\end{quote}

\chapter{Des logiciels libres dans le
terminal}\label{des-logiciels-libres-dans-le-terminal-1}

\section{ImageMagick}\label{imagemagick}

\section{FFmpeg}\label{ffmpeg}

ffmpeg -i input\_file.mp4 -acodec copy -vcodec copy -f mov output.mov

for file in *.Rdata; do mv ``\(file" "\)\{file/.Rdata/\_1.Rdata\}``;
done

https://www.gnu.org/software/make/manual/html\_node/File-Name-Functions.html
